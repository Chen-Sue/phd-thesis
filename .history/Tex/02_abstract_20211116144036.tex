\setlength{\headheight}{26pt}
\maketitle %中文封面
\MAKETITLE %英文封面
\makedeclaration % 作者声明

\intobmk\chapter*{\texorpdfstring{摘\quad 要}{摘要}}
% 研究目的、内容、方法、成果和结论
\setcounter{page}{1}
\pagenumbering{Roman}
时间序列分析是指从时间序列数据中挖掘有用信息的统计技术。随着机器学习的发展,尤其是具备长短期记忆单元的神经网络(Long Short-Term Memory Recurrent Neural Network,简称LSTM-RNN)以及一维卷积神经网络(One Dimensional Convolutional Neural Networks,简称1DCNN),基于机器学习的方法处理时间序列数据取得了一些成果。鉴于部分时间序列数据具有非平稳性、非高斯性和非线性等特征,基于机器学习的分析方法从这种类型的时间序列数据中挖掘有价值的信息仍然具有挑战性。本论文基于机器学习方法,根据数据集的复杂度选择了几种具备代表性的时间序列数据,试图构建不同输入和输出时间窗口长度探索时间序列数据的特征。本论文研究工作的内容和结论如下:
\begin{enumerate}

  \item[(1)] \textbf{利用神经网络预测太阳黑子活动}。太阳黑子是太阳局部强磁场活动在太阳光球表面上产生的黑色斑块,是太阳活动强弱的重要指标。太阳黑子活动影响着地球气候、人类空间活动等。针对第25太阳周的峰值与历史太阳周相比增强还是减弱,目前已有的研究并未得出统一的结论。传统意义上基于物理机制的模型被人为简化,因此这类模型难以准确捕获太阳黑子时间序列趋势。这里基于月均太阳黑子数量和面积,构建不同的输入时间窗口长度(72个月、132个月和264个月)和输出时间窗口长度(1个月和72个月),利用LSTM-RNN、1DCNN、具备LSTM神经元和一维卷积层的神经网络(Long Short-Term Memory and One Dimensional Convolutional Neural Networks,简称LSTM-1DCNN)捕获太阳黑子时间序列趋势。最终发现,三种神经网络算法均能捕获到太阳黑子活动,且LSTM-1DCNN的预测性能略优于LSTM-RNN和1DCNN。基于LSTM-1DCNN的预测结果发现,第25太阳周太阳黑子数峰值为132.86(出现时间2024年12月),太阳黑子占比面积峰值为1469.01(出现时间2025年3月)。两者结果显示,第25太阳周的峰值跟第24太阳周基本持平。

  \item[{(2)}] \textbf{利用机器学习预测龙子祠泉流量变化}。地下水过度开采导致井水干枯、淡水区的水量减少、地面沉降等现象。泉流量是地下水到地表水过渡的重要指标,反映了含水层中水位的动态变化。我们这里选取龙子祠泉作为研究对象,收集了该泉域的泉流量和周围九个区域的降水量,在设置不同输入和输出时间窗口长度(分别设置为1至4个月)下,基于机器学习的分析方法预测泉流量。这些方法包括LSTM-RNN、1DCNN、LSTM-1DCNN、支持向量机(Support Vector Regression,简称SVR)、线性回归(Linear Regression,简称LR)、随机森林(Random Forest,简称RF)、决策树(Decision Tree,简称DT)、K近邻(k-Nearest Neighbor,简称KNN)。总体来看,这些方法都适合预测未来泉流量变化;增加输入时间窗口长度,算法的预测能力会逐渐降低;增加输出时间窗口长度,算法的性能会出现一定幅度的下降。进一步研究发现,仅仅利用历史1个月的泉流量就能精确预测未来1个月龙子祠的泉流量。
  
  \item[(3)] \textbf{利用机器学习对南加州地区进行中期地震预报}。有效地预报强震能够减小人员伤亡和经济损失。地震预报包括时间、地点、震级、可能发生的概率这几个要素。根据预报时间的长短,地震预报可分为长期、中期和短期预报。短期预报不可控因素太多,这里不予以考虑;长期预报需要较长时间的观测资料,目前数据时长仍旧不够。因此,地震中期预报成为关注的重点。这里数据来源于美国南加州地区地震目录,该目录记录时间较长(\sim 90年),完备性较好,且南加州地区的地震活动较为频繁。从地震目录中计算了16种不同的地震因子,采用LSTM-RNN、SVR、LR、RF、DT、KNN、梯度提升回归(Gradient Boosting Regression Tree,简称GBRT)、极端随机森林回归(Extra Trees Regressor,简称ETR)这几种方法,对区域内可能发生的最大震级进行了中期预报。从机器学习算法的表现来看,多数算法均出现了过拟合问题,即训练集的误差很小,而测试集的误差很大。对于DT、KNN和ETR,训练集的误差甚至达到了0,而且所有的训练集都处于置信区间内,即算法能够拟合训练集中的所有特征(包括噪声)。整体来看,本研究中机器学习对南加州地区的地震中期预报是过拟合的,算法预测的结果极易受到数据集扰动的影响,可能是因为输入数据中遗漏了未知的重要变量或需要更长时间的数据集。

  综合以上三个方向,发现利用机器学习探索时间序列数据时,输入和输出时间窗口长度会影响算法的性能:适当的时间窗口长度有助于提高算法性能;增加输出时间窗口长度会降低算法性能。机器学习分析时间序列数据时,需要考虑到数据集的完备性,还要找到合适的输入特征。

\end{enumerate}
\keywords{机器学习,时间序列,泉流量,太阳黑子,中期震级预报}


\intobmk\chapter*{Abstract}
% 研究目的、内容、方法、成果和结论。
Time series analysis is a statistical technique for mining useful information from time series data. With the progress of machine learning (ML), especially the long short term memory recurrent neural network (LSTM-RNN) and one dimensional convolutional neural networks (1DCNN), there are achieving some results dealing with time series data based on ML. However, time series data owns characteristics of non-stationary, non-Gaussian and nonlinear, thus it is difficult to extract information from time series data by ML. Based on ML, we select several time series data, and try to construct different input and output time window length to explore time series data. The contents and conclusions of the research works are as follows:
\begin{enumerate}

  \item[(1)] \textbf{Prediction of sunspot activity by neural network}. Sunspots are black patches produced on the surface of the solar photosphere by the strong local magnetic activity of the Sun, and an important indicator of the strength of solar activity. Sunspot activity will affect earth's climate, human space activities and so on. There are still no unified conclusion on whether the peak value of the cycle 25 shows an upward or downward trend compared with the historical solar cycle. The traditional physical models are simplified, which makes it difficult for accurately capturing the sunspot time series trend. Here, we use the monthly average sunspot number and area, construct different input time window length (72 months, 132 months or 264 months) and output time window length (one month or 72 months), and adopt LSTM-RNN, 1DCNN and LSTM-1DCNN (Long Short-Term Memory and One Dimensional Convolutional Neural Networks). Finally, we find that LSTM-RNN, 1DCNN and LSTM-1DCNN can capture the sunspot activity and the performance of LSTM-1DCNN is slightly better than LSTM-RNN and 1DCNN. Based on the prediction of LSTM-1DCNN, the maximum sunspots' number and area in the cycle 25 predicted by LSTM-1DCNN is 132.86 which occurs in December 2024 and 1469.01 which occurs in March 2025. Both results show that the peak values of cycle 25 are basically the same as cycle 24. 

  \item[(2)] \textbf{Predicting spring discharge in Longzici spring by machine learning}. Over exploitation of groundwater results in dry well water, reduction of water volume in fresh water areas, land subsidence and so on. Spring discharge is an important indicator of the transition from groundwater to surface water, which reflects the dynamic change of aquifer and the normal operation of water flow system in spring area. We choose the Longzici spring. The dataset include the spring discharge and the precipitation at nine surrounding areas, and we set the input and output time window length from 1 to 4 months respectively, and use LSTM-RNN, 1DCNN, long short term memory and one dimensional convolutional neural networks (LSTM-1DCNN), support vector regression (SVR), linear regression (LR), random forest (RF), decision tree (DT), and k-nearest neighbor (KNN) to simulate the trend of spring discharge.
  Overall, these methods are suitable for predicting the various of spring discharge. As for the input window length, the input time window length from one to four month will not improve the performance of the models; For the output windows, it ranges from 1 to 4 months, which will slightly reduce the performance of the model. Otherwise, we find that historical spring discharge for one month can accurately predict the future spring discharge of the Longzici.
  
  \item[(3)] \textbf{Medium term earthquake prediction in Southern California by machine learning}. Effective forecasting of powerful earthquakes can reduce casualties and economic damage. Earthquake prediction involves time, location, magnitude and probability of occurrence. According to time span, earthquake prediction can be divided into long-term, medium-term and short-term prediction. For short-term prediction, there are too many uncontrollable factors; For long-term prediction, the observation dataset are required at least hundreds of years. Our observations duration are still insufficient. Therefore, medium-term earthquake prediction becomes the focus of our study. The seismic dataset comes from the earthquake catalog of Southern California, USA. This catalog is an ideal database because it has been recoding for a long time (\sim 90 years) with good completeness, and seismic activity is more frequent in the Southern California region. From the earthquake catalog, we calculate 16 different earthquake factors and use several ML methods, namely LSTM-RNN, SVR, LR, RF, DT, KNN, gradient boosting regression tree (GBRT) and extra trees regression (ETR), to predict the maximum earthquake magnitude that may occur in the region within the medium term. From the performance of all models, they have a large degree of over fitting. That is, the error of the training set is very small, while the error of the testing set is very large. For DT, KNN and ETR models, the error of the training set even reaches 0 and within the scope of the training data set is in the confidence interval. That is, the model can fit all the features of the training set, including noises. It can be seen from these results that ML methods appears over fitting to explore the medium-term earthquake prediction in Southern California in this study. The possible reasons are followings: We miss some important factors; We need to collect more earthquake catalog.

  Combining these three directions, we find that the input and output time window length affect the performance of the model when exploring time series data using machine learning. Choosing the right rather input time window length helps to improve the model performance. Increasing the output time window decreases the model performance. Machine learning needs to consider the completeness of the dataset and finding the appropriate input features when exploring time series in order to explore the target task.

\end{enumerate}
\KEYWORDS{Machine Learning, Time Series, Spring Discharge, Sunspot, Earthquake Prediction}
