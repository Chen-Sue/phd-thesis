\chapter{总结与展望}\label{chap:conclusion}

% 结论是对论文主要研究结果、论点的提炼与概括,应精炼、准确、完整,使读者看后能全面了解论文的意义、目的和工作内容。
% 结论应包括论文的核心观点,主要阐述作者的创造性工作及所取得的研究成果在本领域中的地位、作用和意义,交代研究工作的局限,提出未来工作的意见或建议。

本文围绕着机器学习方法的应用展开。因为机器学习能够较为精确地从复杂系统中提取关键的信息,因此受到了国内外越来越多的学术界和工业界的关注。基于机器学习在空间结构数据和时间序列数据中的优异性能,本文主要基于地下水、太阳黑子、地震数值模拟等方面展开研究。针对“地下水”问题,本文提出了多种机器学习模型预测未来不同时期龙子祠泉的泉水变化趋势。针对“太阳黑子”,论文基于不同种类的神经网络开展了识别未来不同时间窗口的太阳黑子数量和面积,重点关注第25太阳周的太阳黑子峰值。针对“地震数值模拟”,论文选取了南加州地区作为研究对象,预测未来最大震级。本研究分以下几个方面展开:

\begin{enumerate}
    \item \textbf{利用机器学习预测龙子祠泉流量变化}。目前,人类经济活动致使全球气候变得愈加不稳定,人口的增加又迫使生活用水和农业用水需求的激增。全球大部分未冻结的淡水都存储在地下含水层中,地下水的可持续利用有助于人类经济和社会的稳定发展。然而,近些年来,地下水过度开采,井水干枯、溪流和湖泊等淡水区的水量减少、抽水成本日益增加、地面沉降、水配给不足等现象时有发生。泉流量是地下水到地表水过渡的指标之一,反映了含水层的动态变化,关于大片区域的水流系统。能够准确预知泉流量的变化,对于优化泉水的利用效率,保持合理的农业用水,有助于整个水源的可持续管理。泉流量是一系列动态变化的结果,整个泉域可看作是一个非线性的复杂系统,因为地下水流途径错综复杂。一般来讲,对于泉流量的定量分析需要考虑以下条件:初始泉流量、区域降雨量、地下水开采量、入渗、地表径流、蒸散、地下水补给、土壤水分、侧向水流至蓄水层、含水层和下伏含水层之间的渗漏以及蓄水层中蓄水量的变化等,这些因素与水文地质特征、地貌特征、土地利用、土地覆盖、人为开采和气候条件等息息相关。这里我们使用的是龙子祠泉的数据,具体包括龙子祠泉流量、其周围九个区域(即化乐、克城、山头、一平垣、台头、光华、河底、双风渰、关王庙)的降雨量。需要说明的是,尽管有多种因素会影响龙子祠泉流量,但这里我们只选取了两种因素,即已观测到的泉流量和泉域降水量,因为这两种数据相较于其他因素更好获取和采集。龙子祠泉具有喀斯特地貌的泉域,地下水和地表水对可溶性岩石腐蚀与沉淀,侵蚀与沉积,以及重力崩塌、坍塌、堆积等作用形成的地貌。在溶洞中,地下管道多,水流会形成一个不断扩大的循环网。理论上,利用平衡方程模型可估计未来泉流量。但在实际应用过程中,会经常简化分析。我们基于不同的神经网络在预测范围不超过4个月的情况下预测泉流量并进行比较分析。
    
    经过分析后,我们将输入时间窗口设置分别为3个月、4个月、5个月、6个月,将输入窗口分别设置为1个月、2个月、3个月、4个月。将不同时间窗口的输入数据和输出数据喂给不同架构的机器学习模型,这些模型包括LSTM-RNN,1DCNN,LSTM+1DCNN,SVR(Support Vector Regression),LR(Linear Regression),RF(Random Forest),DT(Decision Tree),KNN($k$-Nearest Neighbor)。
    
    若输入窗口为3个月:当输出窗口为1个月时,RF表现最好,MSE值为0.0010,RMSE值为0.0313;当输出窗口为2个月时,RF和LSTM-RNN两者表现最好,MSE值均为0.0013,RMSE值均为0.0363;当输出窗口为3个月时,RF表现最好,MSE值均为0.0016,RMSE值均为0.0405;当输出窗口为4个月时,SVR表现最好,MSE值均为0.0020,RMSE值均为0.0448。
    
    若输入窗口为4个月:当输出窗口为1个月时,RF表现最好,MSE值为0.0010,RMSE值为0.0319;当输出窗口为2个月时,RF表现最好,MSE值均为0.0015,RMSE值均为0.0392;当输出窗口为3个月时,RF表现最好,MSE值均为0.0019,RMSE值均为0.0437;当输出窗口为4个月时,LSTM-RNN表现最好,MSE值均为0.0021,RMSE值均为0.0459。
  
    若输入窗口为5个月:当输出窗口为1个月时,RF表现最好,MSE值为0.0010,RMSE值为0.0317;当输出窗口为2个月时,RF表现最好,MSE值均为0.0015,RMSE值均为0.0386;当输出窗口为3个月时,RF表现最好,MSE值均为0.0015,RMSE值均为0.0424;当输出窗口为4个月时,RF表现最好,MSE值均为0.0022,RMSE值均为0.0464。
    
    若输入窗口为6个月:当输出窗口为1个月时,RF表现最好,MSE值为0.0011,RMSE值为0.0325;当输出窗口为2个月时,RF表现最好,MSE值均为0.0015,RMSE值均为0.0380;当输出窗口为3个月时,RF表现最好,MSE值均为0.0019,RMSE值均为0.0436;当输出窗口为4个月时,RF表现最好,MSE值均为0.0024,RMSE值均为0.0487。
  
    总体来看,8类不同模型的性能评价指标(MSE和RMSE)都偏小,因此几种方法都适合预测未来泉流量变化。尽管如此,最佳的预测模型是RF,其次为LSTM-RNN和SVR;最差的预测模型是1DCNN,其次为KNN、DT和LR。这说明在小数据集上,神经网络的预测效果并一定比传统的机器学习方法好。就输入窗口而言,3个月到6个月的滞后不会提高模型的预测能力,这在某种程度上说明了输入数据存在一定程度上的冗余,这些冗余信息会在拟合过程中被忽视。就输出窗口而言,1个月到4个月不等的预测会稍微降低模型的性能。
  
  
    \item \textbf{利用神经网络预测太阳黑子变化}。太阳活动时刻在发生,太阳黑子是太阳活动现象过程中很容易观测到的特征之一。太阳黑子的产生是由太阳内部产生的强磁场导致的。太阳黑子变化会影响着日光层环境、地球气候、人类空间活动等。理解太阳活动如何随着时间变化是许多科研工作者关注的重点之一。由于太阳黑子的易观测性,目前有关太阳黑子记录最长可达400多年,而且太阳黑子能够反映太阳活动的剧烈程度,这位理解太阳活动机制奠定了基础。但太阳黑子的时间序列数据具有非平稳性、非高斯性、非线性等特征,因此预测太阳黑子变化始终是一个挑战性课题。即使目前已有研究利用年平均或月平均太阳黑子数量或面积预测未来太阳黑子变化,但针对未来太阳周(例如第25太阳周)的峰值与过去太阳周的峰值相比上升还是下降,目前的研究并未统一。传统意义上的基于物理机制的模型和统计模型在预测太阳黑子变化过程中会刻意简化模型,即假设时间序列是由线性过程产生的,从而难以准确得捕获非高斯性、非线性的太阳黑子时间序列关系。这里我们采用月平均太阳黑子数量和面积分别进行预测和对照分析。采用的方法是不同结构的神经网络,分别为LSTM-RNN、1DCNN和LSTM+1DCNN。类似于泉流量的预测过程,这里我们也采纳了不同的输入时间窗口和输出时间窗口。输入窗口我们选择了1个月、12个月(1年)、72个月(6年)、132个月(11年)、264个月(22年)、396个月(33年)、528个月(44年);输出窗口我们选择了1个月、2个月、3个月、6个月、12个月(1年)、72个月(6年)、132个月(11年)。
    
    我们首先试验未来一个月太阳黑子数的变化情况。对于LSTM-RNN而言,输入节点数在1、12、72、132、264均表现出较好的拟合性能,测试集中MSE值为$\sim$0.0035,RMSE值为$\sim$0.0600;对于1DC而言,因网络结构的限制,输入节点数设为72、132、264、396、528,在输入节点为72、132、264均表现出较好的拟合性能,测试集中MSE值为$\sim$0.0058,RMSE值为$\sim$0.0740;对于LSTM+1DCNN而言,在输入节点为72、132、264同样表现出较好的拟合性能,测试集中MSE值为$\sim$0.0035,RMSE值为$\sim$0.0600。这里可以看出,预测未来一个月的太阳黑子数时,LSTM-RNN和LSTM+1DCNN性能最佳。
  
    这里,我们选取132作为输入窗口大小,输出节点数分别为1、2、3、6、12、72、132。针对LSTM-RNN,输出节点为1时,性能最佳;节点数为其他值时,性能欠佳。针对1DCNN,输出节点为1、2、3、6、12时,均表现出较好的拟合性能,测试集中MSE值为$\sim$0.0060,RMSE值为$\sim$0.0800。针对LSTM+1DCNN,输出节点为1、2、3、6、12时,也都表现出较好的拟合性能,测试集中MSE值为$\sim$0.0045,RMSE值为$\sim$0.0650。整体来看,LSTM+1DCNN的性能优于了LSTM-RNN和1DCNN,而且随着输出节点的增加,模型性能呈现下降趋势。最终发现,利用1DCNN预测的第25太阳周最大MSSN为157.80(发生在2025年1月),利用LSTM+1DCNN预测的第25太阳周最大MSSN为149.22(发生在2025年3月)。经过进一步试验,我们发现LSTM-RNN、1DCNN和LSTM+1DCNN模型在三层时为最优的。结果显示,第25太阳周的峰值跟第24太阳周基本持平。
  
    同样地,我们试验未来一年太阳黑子面积的变化情况,发现LSTM+1DCNN和1DCNN的性能优于了LSTM-RNN和。而且随着层数的增加,模型性能并未出现大幅度提升。在预测未来11年太阳黑子变化时,我们发现利用1DCNN预测第25太阳周的最大MSSA为1861.14(发生在2023年6月),利用LSTM+1DCNN预测第25太阳周的最大MSSA为1496.42(发生在2023年3月)。结果显示,第25太阳周的峰值跟第24太阳周基本持平。跟太阳黑子数预测结果相比,在峰值出现时时间上还存在一定的差距。
    
    \item \textbf{机器学习对南加州地区的地震中期预报}。大震发生很可能会导致严重的人员伤亡和经济损失,影响社会经济的持续发展。为了减小这些损失,预测强震就显得尤为重要。地震预测一般会涉及到几大要素:时间、地点、震级、可能发生的概率。根据预测的时间长度,地震预测分为长期、中期和短期预测。其中,短期预测不可控因素太多,这里不予以考虑。长期预测需要的观测资料年限至少上百年,目前我们的数据时长仍旧欠缺。因此,地震中期预测成为我们关注的重点。我们的地震数据来源于美国南加州地区地震目录。从地震目录中我们计算了16种不同的地震因子,采用不同机器学习方法(LSTM-RNN、SVR、LR、RF、DT、KN、GBR、ETR)对区域内可能发生的最大震级进行中期预测。从所有模型的表现来看,它们均出现了很大程度的过拟合现象。从这些结果可以看出,机器学习对南加州地区的地震中期预报是失败的。可能的原因有:对于本数据集而言,机器学习模型仍然过于复杂;地震目录本身存在噪音;地震的自组织性导致数值预测本身不可靠。

\end{enumerate}