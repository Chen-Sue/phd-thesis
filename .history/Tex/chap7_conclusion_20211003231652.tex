\chapter{总结与展望}\label{chap:conclusion}

% 结论是对论文主要研究结果、论点的提炼与概括,应精炼、准确、完整,使读者看后能全面了解论文的意义、目的和工作内容。
% 结论应包括论文的核心观点,主要阐述作者的创造性工作及所取得的研究成果在本领域中的地位、作用和意义,交代研究工作的局限,提出未来工作的意见或建议。

论文围绕着机器学习方法的应用展开。因为机器学习能够较为精确地从复杂系统中提取关键的信息,因此受到了国内外学术界和工业界越来越多的关注。基于机器学习在空间结构数据和时间序列数据中的优异性能,论文主要基于地下水、太阳黑子、地震数值模拟等方面展开研究。针对“地下水”问题,论文提出了多种机器学习模型预测未来不同时期龙子祠泉的泉水变化趋势;针对“太阳黑子”,论文基于不同种类的神经网络开展了未来不同时间窗口的太阳黑子数量和面积,重点关注第25太阳周的太阳黑子峰值;针对“地震数值模拟”,论文选取了南加州地区作为研究对象,预测未来最大震级。从以下几个方面展开:

\begin{enumerate}
    \item \textbf{利用机器学习预测龙子祠泉流量变化}。目前,人类经济活动致使全球气候变得愈加不稳定,人口的增加又迫使生活用水和农业用水需求的激增。全球大部分未冻结的淡水都存储在地下含水层中,地下水的可持续利用有助于人类经济和社会的稳定发展。然而,近些年来,地下水过度开采,井水干枯、溪流和湖泊等淡水区的水量减少、抽水成本日益增加、地面沉降、水配给不足等现象时有发生。泉流量是地下水到地表水过渡的指标之一,反映了含水层的动态变化。准确预知泉流量的变化,优化泉水的利用效率,保持合理的农业用水,有助于整个水源的可持续管理。泉流量是一系列动态变化的结果,整个泉域可看作是一个非线性的复杂系统,因为地下水流途径错综复杂。一般来讲,对于泉流量的定量分析需要考虑水文地质特征、地貌特征、土地利用、土地覆盖、人为开采和气候条件等。研究对象为龙子祠泉,数据涵盖了龙子祠泉流量、其周围九个区域(即化乐、克城、山头、一平垣、台头、光华、河底、双风渰、关王庙)的降水量。龙子祠泉是具有喀斯特地貌的泉域,地下水和地表水对可溶性岩石腐蚀与沉淀,侵蚀与沉积以及重力崩塌、坍塌、堆积等作用形成的地貌。在溶洞中,地下管道多,水流会形成一个不断扩大的循环网。理论上,利用平衡方程模型可估计未来泉流量。但在实际应用过程中,会经常简化分析。
    
    论文基于不同的机器学习模型在预测范围不超过4个月的情况下预测泉流量并进行比较分析。经过分析后,将输入输和输出时间窗口都分别设置为1至4个月。将不同时间窗口的输入数据和输出数据喂给不同的机器学习模型,这些模型包括LSTM-RNN、1DCNN、LSTM-1DCNN、SVR、LR、RF、DT、KNN。

    \begin{enumerate}
      \item \textbf{输出窗口为一个月}。当输入时间窗口为一个月$I_{-1}$时,LSTM-1DCNN拟合指标MSE$=\SI{0.0009}{m^{3}/s}$, RMSE$=\SI{0.0297}{m^{3}/s}$,预测2019年1月的泉流量为$\SI{2.97}{m^{3}/s}$;当输入时间窗口为两个月$I_{-2}$时,1DCNN拟合指标MSE$=\SI{0.0008}{m^{3}/s}$, RMSE$=\SI{0.0288}{m^{3}/s}$,预测2019年1月的泉流量为$\SI{2.92}{m^{3}/s}$;当输入时间窗口为三个月$I_{-3}$时,RF拟合指标MSE$=\SI{0.0010}{m^{3}/s}$,RMSE$=\SI{0.0313}{m^{3}/s}$,预测2019年1月的泉流量为$\SI{2.99}{m^{3}/s}$;当输入时间窗口为四个月$I_{-4}$时,RF拟合指标MSE$=\SI{0.0010}{m^{3}/s}$,RMSE$=\SI{0.0319}{m^{3}/s}$,预测2019年1月的泉流量为$\SI{2.99}{m^{3}/s}$。所有模型在预测2019年1月的泉流量时,最大差距为$\SI{0.06}{m^{3}/s}$。
      \item \textbf{输出窗口为两个月}。当输入时间窗口为一个月$I_{-1}$时,1DCNN拟合指标MSE$=\SI{0.0012}{m^{3}/s}$,RMSE$=\SI{0.0352}{m^{3}/s}$,预测2019年1月的泉流量为$\SI{2.93}{m^{3}/s}$,2019年2月的泉流量为$\SI{2.94}{m^{3}/s}$;当输入时间窗口为两个月$I_{-2}$时,SVR拟合指标MSE$=\SI{0.0013}{m^{3}/s}$,RMSE$=\SI{0.0360}{m^{3}/s}$,预测2019年1月的泉流量为$\SI{2.98}{m^{3}/s}$,预测2019年2月的泉流量为$\SI{2.01}{m^{3}/s}$;当输入时间窗口为三个月$I_{-3}$时,RF拟合指标MSE$=\SI{0.0013}{m^{3}/s}$,RMSE$=\SI{0.0363}{m^{3}/s}$,预测2019年1月的泉流量为$\SI{2.99}{m^{3}/s}$,2019年2月的泉流量为$\SI{2.93}{m^{3}/s}$;当输入时间窗口为四个月$I_{-4}$时,RF拟合指标MSE$=\SI{0.0014}{m^{3}/s}$,RMSE$=\SI{0.0373}{m^{3}/s}$,预测2019年1月的泉流量为$\SI{3.05}{m^{3}/s}$,预测2019年2月的泉流量为$\SI{3.07}{m^{3}/s}$。
      \item \textbf{输出窗口为三个月}。当输入时间窗口为一个月$I_{-1}$时,LSTM-1DCNN拟合指标MSE$=\SI{0.0015}{m^{3}/s}$,,RMSE$=\SI{0.0389}{m^{3}/s}$,预测2019年1月的泉流量为$\SI{2.98}{m^{3}/s}$,2019年2月的泉流量为$\SI{2.98}{m^{3}/s}$,2019年3月的泉流量为$\SI{2.89}{m^{3}/s}$;当输入时间窗口为两个月$I_{-2}$时,SVR拟合指标MSE$=\SI{0.0013}{m^{3}/s}$,RMSE$=\SI{0.0360}{m^{3}/s}$,预测2019年1月的泉流量为$\SI{2.98}{m^{3}/s}$,2019年2月的泉流量为$\SI{2.91}{m^{3}/s}$,2019年3月的泉流量为$\SI{2.90}{m^{3}/s}$;当输入时间窗口为三个月$I_{-3}$时,LSTM-RNN拟合指标MSE$=\SI{0.0015}{m^{3}/s}$,RMSE$=\SI{0.0390}{m^{3}/s}$,预测2019年1月的泉流量为$\SI{2.30}{m^{3}/s}$,2019年2月的泉流量为$\SI{2.94}{m^{3}/s}$,2019年3月的泉流量为$\SI{2.86}{m^{3}/s}$;当输入时间窗口为四个月$I_{-4}$时,RF拟合指标MSE$=\SI{0.0017}{m^{3}/s}$,RMSE$=\SI{0.0414}{m^{3}/s}$,预测2019年1月的泉流量为$\SI{3.06}{m^{3}/s}$,2019年2月的泉流量为$\SI{3.07}{m^{3}/s}$,2019年3月的泉流量为$\SI{2.99}{m^{3}/s}$。
      \item \textbf{输出窗口为四个月}。当输入时间窗口为四个月时,SVR拟合指标MSE$=\SI{0.0020}{m^{3}/s}$,RMSE$=\SI{0.0448}{m^{3}/s}$,预测2019年1月的泉流量为$\SI{2.9992}{m^{3}/s}$,2019年2月的泉流量为$\SI{2.9144}{m^{3}/s}$,2019年3月的泉流量为$\SI{2.8312}{m^{3}/s}$,2019年4月的泉流量为$\SI{2.7752}{m^{3}/s}$;当输入时间窗口为四个月时,RF拟合指标MSE$=\SI{0.0022}{m^{3}/s}$,RMSE$=\SI{0.0468}{m^{3}/s}$,预测2019年1月的泉流量为$\SI{3.0683}{m^{3}/s}$,2019年2月的泉流量为$\SI{3.0132}{m^{3}/s}$,2019年3月的泉流量为$\SI{2.9706}{m^{3}/s}$,2019年4月的泉流量为$\SI{2.9250}{m^{3}/s}$;当输入时间窗口为五个月时,RF拟合指标MSE$=\SI{0.0022}{m^{3}/s}$,RMSE$=\SI{0.0464}{m^{3}/s}$,预测2019年1月的泉流量为$\SI{3.0646}{m^{3}/s}$,2019年2月的泉流量为$\SI{3.0279}{m^{3}/s}$,2019年3月的泉流量为$\SI{2.9448}{m^{3}/s}$,2019年4月的泉流量为$\SI{2.9055}{m^{3}/s}$;当输入时间窗口为六个月时,RF拟合指标MSE$=\SI{0.0024}{m^{3}/s}$,RMSE$=\SI{0.0487}{m^{3}/s}$,预测2019年1月的泉流量为$\SI{3.0739}{m^{3}/s}$,2019年2月的泉流量为$\SI{3.0337}{m^{3}/s}$,2019年3月的泉流量为$\SI{2.9568}{m^{3}/s}$,2019年4月的泉流量为$\SI{2.8774}{m^{3}/s}$。
    \end{enumerate}
  
    总体来看,8类不同模型的性能评价指标(MSE和RMSE)都偏小,因此几种方法都适合预测未来泉流量变化,最佳的预测模型是RF。这说明在小数据集上,神经网络的预测效果并一定比传统的机器学习方法好。就输入窗口而言,1个月到4个月的滞后不会提高模型的预测能力,这在某种程度上说明了输入数据存在一定程度上的冗余,这些冗余信息会在拟合过程中被忽视。随着输出时间窗口的增加,模型的性能会出现一定幅度的下降。
  
  
    \item \textbf{基于神经网络预测太阳黑子变化}。太阳黑子是太阳活动现象过程中很容易观测到的特征之一。太阳黑子的产生是由太阳内部产生的强磁场导致的。太阳黑子变化会影响着日光层环境、地球气候、人类空间活动等。理解太阳活动如何随着时间变化是许多科研工作者关注的重点之一。由于太阳黑子的易观测性,目前有关太阳黑子记录最长可达400多年,而且太阳黑子能够反映太阳活动的剧烈程度,这为理解太阳活动机制奠定了基础。但太阳黑子的时间序列数据具有非平稳型、非高斯型、非线性等特征,因此预测太阳黑子变化始终是一个挑战性课题。即使目前已有研究利用年平均或月均太阳黑子数量或面积预测未来太阳黑子变化,但针对未来太阳周(例如第25太阳周)的峰值与过去太阳周的峰值相比上升还是下降,目前的研究并未统一。传统意义上的基于物理机制的模型和统计模型在预测太阳黑子变化过程中会刻意简化模型,即假设时间序列是由线性过程产生的,从而难以准确捕获太阳黑子时间序列关系。这里采用月均太阳黑子数量和面积分别进行预测和对照分析。采用不同结构的神经网络,分别为LSTM-RNN、1DCNN和LSTM-1DCNN。类似于泉流量的预测过程,这里也采纳了不同的输入时间窗口和输出时间窗口。输入窗口选择了72个月(6年)、132个月(11年)、264个月(22年);输出窗口选择了1个月和72个月。
    
    \begin{enumerate}
      \item \textbf{讨论输出时间窗口为1个月的太阳黑子强度}。
      \begin{itemize}
        \item \textbf{太阳黑子数}。历史72个月的太阳黑子数量作为时间窗口所得到的模型是最优的。当输入时间窗口为72个月时,LSTM-RNN的拟合指标MSE$=\SI{0.0031}{}$,RMSE$=\SI{0.0554}{}$,预测2021年9月的太阳黑子数为$\SI{37.93}{}$;1DCNN的拟合指标MSE$=\SI{0.0041}{}$,RMSE$=\SI{0.0640}{}$,预测2021年9月的太阳黑子数为$\SI{31.66}{}$;LSTM-1DCNN的拟合指标MSE$=\SI{0.0030}{}$,RMSE$=\SI{0.547}{}$,预测2021年9月的太阳黑子数为$\SI{41.15}{}$。
        \item \textbf{太阳黑子面积}。历史132个月的太阳黑子面积作为时间窗口所得到的模型是最优的。当输入时间窗口为132个月时,LSTM-RNN的拟合指标MSE$=\SI{0.0021}{}$,RMSE$=\SI{0.0463}{}$,预测2021年9月的太阳黑子面积为$\SI{274.14}{}$;1DCNN的拟合指标MSE$=\SI{0.0028}{}$,RMSE$=\SI{0.0531}{}$,预测2021年9月的太阳黑子面积为$\SI{359.78}{}$;LSTM-1DCNN的拟合指标MSE$=\SI{0.0021}{}$,RMSE$=\SI{0.0453}{}$,预测2021年9月的太阳黑子面积为$\SI{88.61}{}$。
      \end{itemize}
      \item \textbf{讨论输出时间窗口为72个月的最大太阳黑子强度}。
      \begin{itemize}
        \item \textbf{太阳黑子数}。当输入时间窗口为264个月时,3层最佳的LSTM-1DCNN的拟合指标MSE$=\SI{0.0085}{}$,RMSE$=\SI{0.0920}{}$,预测未来72个月的太阳黑子数最大值为$\SI{151.55}{}$,发生在2023年9月;4层最佳的LSTM-1DCNN的拟合指标MSE$=\SI{0.0082}{}$,RMSE$=\SI{0.0905}{}$,预测未来72个月的太阳黑子数最大值为$\SI{174.71}{}$,发生在2025年1月;5层最佳的LSTM-1DCNN的拟合指标MSE$=\SI{0.0072}{}$,RMSE$=\SI{0.0849}{}$,预测未来72个月的太阳黑子数最大值为$\SI{132.86}{}$,发生在2024年12月。对黑子数而言,第23太阳周最大MSSN出现在2001年9月,其值为238.2;第24太阳周最大MSSN出现在2014年2月,其值为146.1。研究结果显示,第25太阳周的峰值跟第24太阳周基本持平。 
        \item \textbf{太阳黑子面积}。3层最佳的LSTM-1DCNN模型的输入时间窗口为132个月时,拟合指标MSE$=\SI{0.0078}{}$,RMSE$=\SI{0.0884}{}$,预测未来72个月的太阳黑子面积最大值为$\SI{1016.32}{}$,发生在2024年8月;4层最佳的LSTM-1DCNN的拟合指标MSE$=\SI{0.0082}{}$,RMSE$=\SI{0.0905}{}$,预测未来72个月的太阳黑子面积最大值为$\SI{1469.01}{}$,发生在2025年3月;5层最佳的LSTM-1DCNN的拟合指标MSE$=\SI{0.0072}{}$,RMSE$=\SI{0.0849}{}$,预测未来72个月的太阳黑子面积最大值为$\SI{1397.77}{}$,发生在2024年4月。对于太阳黑子面积,第23太阳周MSSA的峰值为2171.7,出现在2001年9月;第24太阳周MSSA的峰值为1439.82,出现在2014年2月。研究结果显示,第25太阳周的峰值跟第24太阳周基本持平。 
      \end{itemize}
    \end{enumerate}
    
    \item \textbf{基于机器学习对南加州地区的地震中期预报}。大震发生很可能会导致严重的人员伤亡和经济损失,影响社会经济的持续发展。为了减小这些损失,预测强震就显得尤为重要。地震预测一般会涉及到几大要素:时间、地点、震级、可能发生的概率。根据预测的时间长度,地震预测分为长期、中期和短期预测。其中,短期预测不可控因素太多,这里不予以考虑。长期预测需要的观测资料年限至少上百年,目前数据时长仍旧欠缺。因此,地震中期预测成为关注的重点。地震数据来源于美国南加州地区地震目录。从地震目录中计算了16种不同的地震因子,采用不同机器学习方法(LSTM-RNN、SVR、LR、RF、DT、KN、GBR、ETR)对区域内可能发生的最大震级进行了以下几种试验。
    \begin{enumerate}
      \item 将研究区域划分为6个不同区块,数据集划分为0.8:0.2,预测未来1年的最大震级;
      \item 基于整个区块,数据集比例0.8:0.2,预测未来一年的最大震级;
      \item 基于整个区块,数据集比例分别为0.8:0.2、0.85:0.15和0.9:0.1,预测未来十年的最大震级。
    \end{enumerate}
    
    从多数模型的表现来看,它们均出现了很大程度的过拟合现象。当基于整个区块,而且划分比例为0.9:0.1时,LSTM-RNN、SVR、LR三种模型均出现了一定程度的欠拟合,RF、GBR、DT、KNN、ETR这几种模型出现了一定程度的过拟合。这种情况下,试验是最优的。但模型很容易受到数据集的影响,即数据集的微小变化对模型的性能会产生巨大的影响,即模型产生了高方差。本研究中机器学习对南加州地区的地震中期预报是失败的。导致这种情况的出现可能因为选取的数据集长度不够或忽略了某些重要的输入特征。

\end{enumerate}