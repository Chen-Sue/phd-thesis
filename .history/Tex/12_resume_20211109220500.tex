\chapter[致谢]{致\quad 谢}\chaptermark{致\quad 谢}

时光如白驹过隙,恍惚间博士生涯已进入尾声阶段,感慨万千。3年前,自己怀着对学术生涯的期待与中国科学院大学结下了缘分。这段路程饱含辛酸与苦楚,但我终不后悔当初的选择。这段博士生涯已深深烙印在我的记忆中,成为我人生中宝贵的财富。回顾这段求学时光,我品尝过无数次的试验失败带来的挫折,但在怀疑自我之后,仍旧振奋精神重新来过。

首先,我想感谢我的导师石耀霖院士。教诲如春风,师恩似海深。三年多的时光,受到了石老师的悉心教导。石老师在科研上给予我最大的支持与关照。在与石老师沟通互动过程中,我深刻体会到了石老师对学术的严谨与执着,对学生的热忱与极力支持,其终身学习的态度令我生畏并让我努力向石老师的精神世界看齐。博士阶段,对于科学研究方法和过程,如何成为一名合格的研究者,石老师都为我倾注了大量心血。

同时,我想感谢的是我的第二指导老师张怀教授。无论在科研还是生活上,张老师是我的良师益友。他始终激励着我,让我从容自信地面对每个挑战,特别是在科研的瓶颈期和焦灼期。

另外,我还想感谢实验室的其他老师、同学和行政人员。感谢周元泽教授对此论文的指导。感谢乔小娟教授跟我开展的关于泉流量研究的合作。同时感激程惠红教授对关于汪荣江老师程序的指导和GMT软件的使用。感谢金一民师兄在程序上提供的交流,王然师兄在\LaTeX{}使用上提供的帮助。感谢郭一村师兄积极交流讨论科研的思路与写论文的方法。感谢实验室其他兄弟姐妹的陪伴。感谢实验室为科研提供的工作环境和人文关怀,让我在体验到温馨的家园感。

当然,家人的支持与理解给了我精神上的援助。感谢我的父亲支撑起了整个家庭,母亲为这个家庭提供了温暖的港湾。有了这些积淀,我才有机会迈入博士队伍。同时需要感激我的姐姐和弟弟。我还要感谢我的男友小灰灰,有了你的鼓励与情感共鸣,让我的人生打开了新的一扇窗。

% syntax: \chapter[目录]{标题}\chaptermark{页眉}
%\thispagestyle{noheaderstyle}%如果需要移除当前页的页眉
%\pagestyle{noheaderstyle}%如果需要移除整章的页眉

\chapter{作者简历及攻读学位期间发表的学术论文与研究成果}

\section*{作者简历:}
\noindent 2011年9月---2015年7月,在青岛理工大学计算机与工程学院获得学士学位。\\
\noindent 2015年9月---2018年7月,在云南大学发展研究院获得硕士学位。\\
\noindent 2018年9月---2021年12月,在中国科学院大学地球与行星科学学院攻读博士学位。

% \noindent
% 工作经历:

\section*{已发表(或正式接受)的学术论文:}
{\setlist[enumerate]{}
\begin{enumerate}[nosep]
  \item Cheng Shu, Qiao Xiaojuan, Shi Yaolin, et al. Machine learning for predicting discharge fluctuation of a karst spring in North China[J]. Acta Geophysica, 2021, 69(1): 257-270.
  \item 程术, 石耀霖, \& 张怀. 基于神经网络预测太阳黑子变化. 中国科学院大学学报[J]. 2021.
  \item 李林芳, 石耀霖, \& 程术. 长短时记忆神经网络在中期地震预报中的探索——以川滇地区为例[J]. 地球物理学报. 2021. 
  \item 石耀霖, 李林芳, \& 程术. 运用lstm神经网络对川滇地区的地震中期预报—回溯性预测2008年汶川Ms8.0地震的探索[J]. 中国科学院大学学报. 2021. 
  \item Cheng Shu, He Fei, Zhang Huai, et al. Machine learning percolation model. arXiv. Under review by Physical Review Research. 2021. 
\end{enumerate}}

% \section*{申请或已获得的专利:}
% (无专利时此项不必列出)

\section*{参加的研究项目及获奖情况:}

\noindent 参加的研究项目:
\begin{table}[htbp]
\footnotesize
\begin{tabular}{cll}
  \toprule
  编号 & 题目 & 类别 \\
  \midrule
  U1839207 & 大数据与地震数值预测探索 & 国家自然科学联合基金 \\
  41725017 & 计算地球动力学 & 国家自然科学基金 \\
  U2039207 & 基于数值模拟的确定性---概率地震危险性分析方法研究 & 国家自然科学基金 \\
  \bottomrule
  \end{tabular}
\end{table}

\noindent 获奖情况:荣获2021学年“三好学生”荣誉称号。

\cleardoublepage[plain] 
% 让文档总是结束于偶数页,可根据需要设定页眉页脚样式,如 [noheaderstyle]

