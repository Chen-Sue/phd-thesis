\section{总结与展望}\label{总结与展望}

% 结论是对论文主要研究结果、论点的提炼与概括,应精炼、准确、完整,使读者看后能全面了解论文的意义、目的和工作内容。
% 结论是最终的、总体的结论,不是正文各章小结的简单重复。
% 结论应包括论文的核心观点,主要阐述作者的创造性工作及所取得的研究成果在本领域中的地位、作用和意义,交代研究工作的局限,提出未来工作的意见或建议。
% 同时,要严格区分自己取得的成果与指导教师及他人的学术成果。

% 在评价自己的研究工作成果时,要实事求是,除非有足够的证据表明自己的研究是“首次”、“领先”、“填补空白”的,否则应避免使用这些或类似词语。

本文围绕着机器学习方法的应用展开。因为机器学习能够较为精确地从复杂系统中提取关键的信息,因此受到了国内外越来越多的学术界和工业界的关注。基于机器学习在空间结构数据和时间序列数据中的优异性能,本文主要基于地下水、太阳黑子、地震数值模拟等方面展开研究。针对“地下水”问题,本文提出了多种机器学习模型预测未来不同时期龙子祠泉的泉水变化趋势。针对“太阳黑子”,论文基于不同种类的神经网络开展了识别未来不同时间窗口的太阳黑子数量和面积,重点关注第25太阳周的太阳黑子峰值。针对“地震数值模拟”,论文选取了南加州地区作为研究对象