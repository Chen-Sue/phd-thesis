%-----------------------------%
%->> Frontmatter
%----------------------------%

%-
%-> 生成封面
%-
\maketitle% 生成中文封面
\MAKETITLE% 生成英文封面

%-
%-> 作者声明
%-
\makedeclaration% 生成声明页

%-
%-> 中文摘要
%-
% 显示在书签但不显示在目录
\intobmk\chapter*{\texorpdfstring{摘\quad 要}{摘要}}
\setcounter{page}{1}% 开始页码
\pagenumbering{Roman}% 页码符号

本文是中国科学院大学学位论文模板ucasthesis的使用说明文档。主要内容为介绍\LaTeX{}文档类ucasthesis的用法,以及如何使用\LaTeX{}快速高效地撰写学位论文。

  论文的摘要是对论文研究内容和成果的高度概括。
  摘要应对论文所研究的问题及其研究目的进行描述,对研究方法和过程进行简单介绍,对研究成果和所得结论进行概括。
  摘要应具有独立性和自明性,其内容应包含与论文全文同等量的主要信息。
  使读者即使不阅读全文,通过摘要就能了解论文的总体内容和主要成果。

  论文摘要的书写应力求精确、简明。
  切忌写成对论文书写内容进行提要的形式,尤其要避免“第 1 章……;第 2 章……;……”这种或类似的陈述方式。

  关键词是为了文献标引工作、用以表示全文主要内容信息的单词或术语。
  关键词不超过 5 个,每个关键词中间用分号分隔。
  
  论文摘要包括中文摘要和英文摘要(Abstract)两部分。论文摘要应概括地反映出本论文的主要内容,说明本论文的主要研究目的、内容、方法、成果和结论。要突出本论文的创造性成果或新见解,不宜使用公式、图表、表格或其他插图材料,不标注引用文献。中文摘要的字数由各学科群分会根据本分会涉及学科专业的特点提出具体要求。英文摘要与中文摘要内容应完全一致。留学生用其他语种撰写学位论文时,中文摘要的字数由学科群分会具体制定,推荐不少于5000字。
摘要最后注明本文的关键词(3~5个)。关键词是为了文献标引工作,从论文中选取出来, 用以表示全文主题内容信息的单词或术语。关键词以显著的字符另起一行并隔行排列于摘要下方,左顶格,关键词间用逗号隔开。英文关键词应与中文关键词对应,首字母应大写。

\keywords{中国科学院大学,学位论文,\LaTeX{}模板}



%-
%-> 英文摘要
%-
\intobmk\chapter*{Abstract}% 显示在书签但不显示在目录

This paper is a help documentation for the \LaTeX{} class ucasthesis, which is  a thesis template for the University of Chinese Academy of Sciences. The main content is about how to use the ucasthesis, as well as how to write thesis efficiently by using \LaTeX{}.

\KEYWORDS{University of Chinese Academy of Sciences (UCAS), Thesis, \LaTeX{} Template}% 英文关键词

